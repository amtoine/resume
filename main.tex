    % FortySecondsCV LaTeX template
% Copyright © 2019-2020 René Wirnata <rene.wirnata@pandascience.net>
% Licensed under the 3-Clause BSD License. See LICENSE file for details.
%
% Please visit https://github.com/PandaScience/FortySecondsCV for the most
% recent version! For bugs or feature requests, please open a new issue on
% github.
%
% Contributors
% ------------
% * ifokkema
% * Bertbk
% * Hespe
%
% Attributions
% ------------
% * fortysecondscv is based on the twentysecondcv class by Carmine Spagnuolo
%   (cspagnuolo@unisa.it), released under the MIT license and available under
%   https://github.com/spagnuolocarmine/TwentySecondsCurriculumVitae-LaTex
% * further attributions are indicated immediately before corresponding code


%-------------------------------------------------------------------------------
%                             ADDITIONAL PACKAGES
%-------------------------------------------------------------------------------
\documentclass[
	a4paper,
	% showframes,
	% vline=2.2em,
	% maincolor=cvgreen,
	% sidecolor=gray!50,
	% sectioncolor=red,
	% subsectioncolor=orange,
	% itemtextcolor=black!80,
	% sidebarwidth=0.4\paperwidth,
	% topbottommargin=0.03\paperheight,
	% leftrightmargin=20pt,
	% profilepicsize=4.5cm,
	% profilepicborderwidth=3.5pt,
	% profilepicstyle=profilecircle,
	% profilepiczoom=1.0,
	% profilepicxshift=0mm,
	% profilepicyshift=0mm,
	% profilepicrounding=1.0cm,
]{fortysecondscv}

% improve word spacing and hyphenation
\usepackage{microtype}
\usepackage{ragged2e}

% uncomment in case you don't want any hyphenation
% \usepackage[none]{hyphenat}

% take care of proper font encoding
\ifxetexorluatex
	\usepackage{fontspec}
	\defaultfontfeatures{Ligatures=TeX}
%	\newfontfamily\headingfont[Path = fonts/]{segoeuib.ttf} % local font
\else
	\usepackage[utf8]{inputenc}
	\usepackage[T1]{fontenc}
%	\usepackage[sfdefault]{noto} % use noto google font
\fi

% enable mathematical syntax for some symbols like \varnothing
\usepackage{amssymb}

% bubble diagram configuration
\usepackage{smartdiagram}
\smartdiagramset{
	% default font size is \large, so adjust to harmonize with sidebar layout
	bubble center node font = \footnotesize,
	bubble node font = \footnotesize,
	% default: 4cm/2.5cm; make minimum diameter relative to sidebar size
	bubble center node size = 0.4\sidebartextwidth,
	bubble node size = 0.25\sidebartextwidth,
	distance center/other bubbles = 1.5em,
	% set center bubble color
	bubble center node color = maincolor!70,
	% define the list of colors usable in the diagram
	set color list = {maincolor!10, maincolor!40,
	maincolor!20, maincolor!60, maincolor!35},
	% sets the opacity at which the bubbles are shown
	bubble fill opacity = 0.8,
}


%-------------------------------------------------------------------------------
%                            PERSONAL INFORMATION
%-------------------------------------------------------------------------------
%% mandatory information
% your name
\cvname{STEVAN Antoine}
% job title/career
\cvjobtitle{Post-graduate engineer}

%% optional information
% profile picture
%\cvprofilepic{res/images/profile.png}

% NOTE: ordering in sidebar will mimic the following order
% date of birth
\cvbirthday{March 29, 1999}
% short address/location, use \newline if more than 1 line is required
\cvaddress{14 chemin du Valon, 17000, Toulouse, France}
% phone number
\cvphone{+33 7 64 42 97 00}
% personal website
\cvsite{amtoine.github.io}{https://amtoine.github.io/public}
% email address
\cvmail{stevan.antoine@gmail.com}
% pgp key
\cvkey{86CFF1AB}{stevan.antoine@gmail.com}
% any other custom entry
%\cvcustomdata{\faFlag}{Chinese}

%-------------------------------------------------------------------------------
%                              SIDEBAR 1st PAGE
%-------------------------------------------------------------------------------
% add more profile sections to sidebar on first page
\addtofrontsidebar{
	% include gosquare national flags from https://github.com/gosquared/flags;
	% naming according to ISO 3166-1 alpha-2 country codes
	\graphicspath{{res/images/flags/}}

	% social network accounts incl. proper hyperlinks
	\profilesection{Social Network}
		\begin{icontable}{2.5em}{1em}
			\social{\faLinkedin}
				{https://www.linkedin.com/in/antoine-stevan/}
				{\underline{My professional page}}
			\social{\faGithub}
				{https://github.com/amtoine}
				{\underline{All my open source projects}}
			\social{\faLinux}
				{https://github.com/goatfiles}
				{\underline{My Linux endeavour}}
		\end{icontable}

	\profilesection{Languages}
        \skill{\flag{FR.png}}{French - native speaker}
        \skill{\flag{GB.png}}{English - fluent}
        \skill{\flag{DE.png}}{German - very good}
        \skill{\flag{JPN.png}}{Japanese - beginner}
        \skill{\includegraphics[scale=.3]{res/images/lsf.png}}{LSF - beginner}
        
	\profilesection{Hard Skills}
		\skill{\faWindows}{
			Linux,
			Arch,
			Windows,
			VMs
		}
		\skill{\faCode}{
			\texttt{python},
			\texttt{rust},
			\texttt{oberon},
			\texttt{LaTeX},
			\texttt{C},
			\texttt{C++},
			\texttt{java},
			processing,
			Assembly,
			Arduino,
			\texttt{CamL},
			\texttt{Racket},
		}
		\skill{\faList}{
			MySQL,
			PostGreSQL,
			Redis
		}
		\skill{\faFlask}{
			JAX,
			PyTorch,
			NumPy,
			Pandas,
			SciPy,
			Scikit-Learn
		}
		\skill{\faTty}{
			\texttt{sh},
			\texttt{bash},
			\texttt{fish},
			\texttt{nushell},
			\texttt{ssh},
			\texttt{git},
			\texttt{docrrr},
		}
		\skill{\faDesktop}{
			Jupyter,
			Google
			Colab,
			Google
			Cloud,
			Cluster
			Computing,
			SpreadSheet,
			Gimp,
			Blender
		}
		\skill{\faCloud}{
			\texttt{markdown},
			\texttt{org},
			\texttt{css},
			\texttt{JavaScript},
			\texttt{html},
		}

	\profilesection{Soft Skills}
    	\begin{sidebarminipage}
    		\chartlabel[lightblue]{Dedicated}
    		\chartlabel[lightblue]{Intrinsically curious}
    		\chartlabel[lightblue]{Calm under pressure}
    		\chartlabel[lightblue]{Versatile}
    		\chartlabel[lightblue]{Fast-learner}
    		\chartlabel[lightblue]{Spontaneous}
    		\chartlabel[lightblue]{Honest}
    		\chartlabel[lightblue]{Open-minded}
    	\end{sidebarminipage}
}


%-------------------------------------------------------------------------------
%                              SIDEBAR 2nd PAGE
%-------------------------------------------------------------------------------
\addtobacksidebar{
	\profilesection{Memberships}
    	\begin{memberships}
    		\membership[4em]{res/images/instadeep.logo.png}
    		{
    		    \href{https://www.instadeep.com/}{\underline{InstaDeep}}, "decision making AI for the enterprise" 
				\newline
				(2022/04 - 2022/08)
    		}
    		\membership[4em]{res/images/sureli.png}
    		{
        		The \href{https://github.com/SuReLI}{\underline{SuReLI lab}}
				\newline
				(2020 - 2021)
				\newline
    		}
    		\membership[3em]{res/images/isae.png}
    		{
    	    	VP of the \textit{\href{https://github.com/iScsc}{Supaero Computer Science Club}}
				\newline
				(2020 - now)
				\newline
    		}
    		\membership[4em]{res/images/innovspace.png}
    		{
    		    The \href{https://sites.google.com/view/innovspace}{\underline{InnovSpace}} of Supaero
				\newline
				(2019 - now)
				\newline
    		}
    	\end{memberships}
}


%-------------------------------------------------------------------------------
%                     TABLE ENTRIES RIGHT COLUMN: FIRST PAGE
%-------------------------------------------------------------------------------
\begin{document}

\makefrontsidebar
\cvsection{About me \& Objectives}
    \cvsubsection{
        Interested in compilation and safe and structured languages.
    }

\cvsection{Working Experience}
% cd\cvsubsection{Recent experiences.}
\begin{cvtable}[4]
	\cvitem
    	{april 2022 -- sept. 2022}
    	{$3^{rd}$ year internship (end of study at Supaero) \newline (5 months)}
    	{InstaDeep, Paris}
    	{
    	    Advance research and help firms with Multi-Objective RL (MORL).
        	\begin{itemize}
        	    \item Explore and benchmark existing MORL algorithms.
        	    \item Propose and implement new techniques using RL research.
        	    \item Give concrete solutions to be deployed as production tools.
        	\end{itemize}
        	 \chartlabel[maincolor]{Multi-Objective RL} \chartlabel[lightblue]{Team work} \chartlabel[lightblue]{Code and ideas coordination}
    	}
	\cvitem
    	{july 2021 -- august 2021}
    	{$2^{nd}$ year internship \newline (7 weeks)}
    	{SuReLI, ISAE-supaero}
    	{
        	Continuation of the previous research topic of my PIR project:
        	\begin{itemize}
        	    \item Further in the disentanglement model \& learning technique.
        	\end{itemize}
        	\chartlabel[maincolor]{Full ML pipeline} \chartlabel[maincolor]{Image processing} \chartlabel[maincolor]{Imitation RL}\newline
        	\chartlabel[lightblue]{Autonomy} \chartlabel[lightblue]{Talking about my work} \chartlabel[lightblue]{Resilience to failure}
    	}
	\cvitem
    	{january 2021 -- july 2021}
    	{\textit{Innovation \& Research Project} (PIR)\newline (5 months)}
    	{ISAE-Supaero}
    	{
    	    First research experience leading to a paper with academic level.\newline
            Title: \textit{Applying Disentanglement for Domain Adaptation to RL}.
            \begin{itemize}
                \item Key questions and techniques of disentanglement.
                \item Short paper submitted to the ISAE-Supaero.
            \end{itemize}
        	\chartlabel[maincolor]{Image processing} \chartlabel[maincolor]{Disentanglement} \chartlabel[maincolor]{Submitting a paper}\newline
        	\chartlabel[lightblue]{Writing communication} \chartlabel[lightblue]{Oral presentation} \chartlabel[lightblue]{Fast learning}
    	}
	\cvitem
    	{nov. 2020 -- april 2021}
    	{Tutor for students \newline (6 months / part-time)}
    	{ISAE-Supaero}
    	{
    	    \begin{itemize}
            	\item Organized support sessions about basic or deeper notions.
            	\item Helped students to graduate their $2^{nd}$ year.
    	    \end{itemize}
        	\chartlabel[lightblue]{Patience}\chartlabel[lightblue]{Popularization skill}\chartlabel[lightblue]{Time Management}
    	}
	\cvitem
    	{october 2020 -- may 2021}
    	{Part-time operator \newline (8 months)}
    	{ISAE-Supaero, InnovSpace}
    	{
    	    \begin{itemize}
            	\item Managed ISAE-Supaero's InnovSpace during my free-time.
            	\item Watched the place, the tools and helped users.
    	    \end{itemize}
        	\chartlabel[maincolor]{Manual tools} \chartlabel[maincolor]{Professional management}\newline
        	\chartlabel[lightblue]{Dedication} \chartlabel[lightblue]{Autonomy}
    	}
\end{cvtable}
\cvsignature

%-------------------------------------------------------------------------------
%                     TABLE ENTRIES RIGHT COLUMN: SECOND PAGE
%-------------------------------------------------------------------------------
\newpage
\makebacksidebar
\cvsubsection{Past experiences}
\begin{cvtable}[2]
	\cvitem
    	{july 2020 -- august 2020}
    	{$1^{st}$ year internship\newline (8 weeks)}
    	{Naval Force 3, La Rochelle, France}
    	{
	        Naval Force 3 is a private shipyard (SME) in La Rochelle.
    	    \begin{itemize}
            	\item First experience in a firm.
            	\item Discovered how it is managed and difficulties due to crisis.
    	    \end{itemize}
        	\chartlabel[maincolor]{Building nautical objects} \chartlabel[maincolor]{Resin techniques}\newline
        	\chartlabel[lightblue]{Managing a workshop} \chartlabel[lightblue]{Going out of comfort zone}
    	}
	\cvitem
    	{sept. 2018 -- june 2019}
    	{Research school project (TIPE) \newline (9 months / part-time)}
    	{Georges Clemenceau, Nantes, France}
    	 {
            First research experience.
    	    \begin{itemize}
            	\item Optimization of the urban space through simulation.
            	\item Learning-based algorithm to facilitate pedestrian traffic.
    	    \end{itemize}
        	\chartlabel[maincolor]{Problem statement} \chartlabel[maincolor]{Evolutionary algorithms}\newline
        	\chartlabel[lightblue]{Autonomy} \chartlabel[lightblue]{Oral communication}
    	}
\end{cvtable}


\cvsection{Education}
% \cvsubsection{Study}
\begin{cvtable}[1.5]
	\cvitem{2019 -- 2022}{ISAE-Supaero}{Toulouse, France}
	{
    	The \textit{Aeronautic \& Space Superior Institute} in France. \newline
        Currently attending the \textit{Data Science \& Decision Making} (SDD) class:
        \begin{itemize}
            \item geometrical, probabilistic and connectionist approaches
            \item commitee-based ML, Deep Learning and RL.
            \item stochastic optimization and evolution.
        \end{itemize}
	}
	\cvitem{2017 -- 2019}{Preparatory Class [MPSI, MP, computer science]}{Georges Clemenceau High School, Nantes}{French class to prepare for highly selective graduate schools. \newline}
	\cvitem{2014 -- 2017}{High school [engineering sciences and computer science]}{Léonce Vieljeux High School, France}{Good results in high schools opens the way to preparatory classes after graduation, which are selective classes.}

\end{cvtable}

%\cvsection{Publications}
%\cvsection{Awards}
\cvsection{Diplomas}
	\cvpubitem{Bachelor's degree}{Bachelor science diploma}
		{}{2020}
	\cvpubitem{TOEFL PBT}{607/677 (B2/C1)}
	    {}{2019}
	\cvpubitem{Graduate schools entrance examination}{Successful examination for the highly selective Mines school bank.}
	    {}{2019}
	\cvpubitem{Baccalauréat (bac)}{Mention: very good}
		{}{2017}

\cvsection{Extra-Curricular Activities}
    \begin{cvtable}
    	\cvitemshort{\includegraphics[scale=0.01]{res/images/arch.png} Arch}{Configuring and building software for my ArchLinux distro.}
    	\cvitemshort{\faCloud ~ Website}{I document my Arch configuration on my websites.}
    	\cvitemshort{\faGamepad ~ Projects}{Coding games in the SCSC or ML algorithms in my free time for fun.}
    	\cvitemshort{\faPaintBrush ~ Epoxy}{An artistic pane out of wood and resin to decorate my place.}
    	\cvitemshort{\faDesktop ~ Computer}{6502 $\mu$-computer, on breadboard, with custom games in assembly.}
    	\cvitemshort{\faDesktop ~ 8-bit CPU}{A CPU out of logic gates with custom assembly.}
    \end{cvtable}
\cvsignature

\end{document}
